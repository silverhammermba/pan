\documentclass[12pt]{article}
\usepackage{amsmath,amssymb,enumerate,mathtools}
\usepackage[margin=1in]{geometry}

\title{CSE 450 Midterm Report:\\
Collaborative Auditing with Differential Correlation}
\author{Maxwell Anselm}
\date{\today}

\begin{document}
\maketitle

From XRay by L\'ecuyer et al., ``a collaborative approach to auditing, in which
users contribute their ads and input topics in an privacy-preserving way is a
promising direction for strengthening robustness against attacks.''

\section{Analysis}

Because this project is primarily based on the existing work of XRay, most
effort up to this point has gone into systematically analyzing every component
of XRay to determine which aspects can be carried through this work, which
must be modified or removed, and which new aspects can be considered.

\subsection{Components}

XRay is composed of three components: a service-specific browser plugin, a
service-specific shadow account manager, and a service-agnostic correlation
engine. The main motivation for this research is the removal of the shadow
account manager. Replacing this manager with a peer-to-peer network would
resolve many of the stated difficulties and limitations of the XRay tool.
However this one change has significant ramifications throughout XRay's
framework, thus motivating a reexamination of every aspect of the tool.

\subsection{Targeting}

XRay is designed to audit three types of advertisement targeting: profiled,
contextual, and behavioral. Profiled and contextual targeting are the simplest
types and are likely to apply to a peer-to-peer model. Behavioral targeting,
however, presents the most difficulty to XRay since it considers a large volume
of user-generated data. This is difficult for the shadow account manager
because the more data considered, the more shadow accounts are needed to
perform the differential correlation.

XRay's major breakthrough is only requiring a logarithmic number of shadow
accounts, rather than the na\"ive exponential number. However this still
constitutes a limitation: auditing just 51 emails still requires 21 shadow
accounts. This can be a serious limitation depending on the breadth of targeting
employed by the service provider and the volume of data generated by the user.
Since XRay considers shadow accounts to be ``generally scarce'' and limited to
``a couple dozen'', they must necessarily restrict their behavioral targeting to
``recent'' data.

With a peer-to-peer model where the shadow accounts are replaced by other real
accounts, this limitation may be unnecessary. If enough peers are present, there
may be no shortage of accounts to compare against. The difficulty then is
finding similar accounts: XRay can simply construct fake accounts with the
desired data, but a peer-to-peer model must work with what is available.

A similar difficulty present in the peer-to-peer situation is collecting a
sufficient volume of ads. XRay can boost the number of ads seen by refreshing
the shadow accounts multiple times in order to quickly see many ads per input.
The peer-to-peer model may have to function with less data than XRay.

\subsection{Overlapping Inputs}\label{olap}

One of the challenges that needed to be overcome by the XRay tool was the
``overlapping inputs'' problem. This problem arose when an ad was targeted
against many different-yet-similar inputs. If these inputs were then randomly
distributed among the shadow accounts, the resulting analysis would incorrectly
label the ad as untargeted. The XRay team solved this problem by grouping inputs
among the shadow accounts according to which ads they appeared with. For
example, if several e-mails all saw the same ad, XRay would group them together
when randomly assigning them to shadow accounts.

Especially for cases like email, this solution is not viable in a peer-to-peer
situation. It is extremely unlikely that two different email users will generate
the exact same input (i.e. email). Thus, from the perspective of XRay, the
peer-to-peer model is {\it always} in the worst case of overlapping inputs:
every ad is targeted at completely distinct inputs randomly distributed among
accounts.

At this point, this appears to be one of the biggest differences between the
peer-to-peer model compared to XRay's shadow accounts. Since XRay crafts the
shadow accounts from duplicate inputs, it has the luxury of treating each email
as a datum. But in a peer-to-peer network where users must compare their inputs,
it is necessary to analyze the content of each email in order to determine which
inputs are similar across accounts.

This problem is not present in simpler services such as Amazon, where each
users' data can be interpreted as a simple binary vector of which items they did
or did not buy. But in unbounded domains such as emails or search queries,
textual similarity analysis may be necessary.

\section{Privacy}

Aside from the topics addressed in the XRay paper, the peer-to-peer approach
introduces a new challenge: how to share advertising information among peers
without comprising users' privacy.

Although methods such as private set intersection were considered, they do not
adequately address the problem space. Specifically, when it comes to
collaborative auditing, the input/output relationships (i.e. the ad targeting
criteria) do not need to be kept private. In fact, making these relationships
public would greatly increase the power of any collaborative system since once
an ad has been successfully audited once, that information can be shared among
the peers.

What needs to be kept private is {\it who has seen which ads}. With XRay, when
you see a possibly sensitive ad, you query the shadow account database to
determine its targeting. With a peer-to-peer network, you would instead need to
communicate to many peers that you saw the ad. If the ad targets sensitive
information, this effectively communicates your sensitive information to your
peers.

The most promising solution to this problem seems to be a TOR-like peer-to-peer
network. For example, requests for information could be routed through a chain
of peers in the network so that no single peer knows who the requester is. This
would also carry the benefit that each peer in the chain would gain further
auditing capabilities through the shared information. This idea requires
concrete development in order to verify its efficacy.

\section{Bounds}

An unfortunate development is that it seems that the mathematical proofs of the
XRay paper may not apply to the peer-to-peer situation. For example the proofs
of the efficiency and accuracy of their central correlation algorithm rely on
the particular structure of their shadow accounts. Although this does not
invalidate their correlation methods for use with a peer-to-peer network, a
proof of the algorithm will need to be crafted from scratch for the new
situation.

Progress in this area will most likely depend on solving the overlapping inputs
problem mentioned in section \ref{olap}. If the service-specific browser plugin
component can reduce an unbounded input space such as email to something bounded
and discrete, it will be much simpler to re-prove the usefulness of XRay's
correlation method in a peer-to-peer setting.

\end{document}
