\documentclass{article}
\usepackage{amsmath,amssymb,enumerate,mathtools}
\usepackage[margin=1in]{geometry}

\title{CSE 450 Midterm Report:\\
Collaborative Auditing with Differential Correlation}
\author{Maxwell Anselm}
\date{\today}

\begin{document}
\maketitle

From XRay by L\'ecuyer et al., ``a collaborative approach to auditing, in which
users contribute their ads and input topics in an privacy-preserving way is a
promising direction for strengthening robustness against attacks.''

\section{Progress}

Because this project is primarily based on the existing work of XRay, most
effort up to this point has gone into systematically analyzing every component
of XRay to determine which aspects can be carried through this work, which
must be modified or removed, and which new aspects can be considered.

\subsection{Components}

XRay is composed of three components: a service-specific browser plugin, a
service-specific shadow account manager, and a service-agnostic correlation
engine. The main motivation for this research is the removal of the shadow
account manager. Replacing this manager with a peer-to-peer network would
resolve many of the stated difficulties and limitations of the XRay tool.
However this one change has significant ramifications throughout XRay's
framework, thus motivating a reexemantion of every aspect of the tool.

\subsection{Targeting}

XRay is designed to audit three types of advertisement targeting: profiled,
contextual, and behavioral. Profiled and contextual targeting are the simplest
types and are likely to apply to a peer-to-peer model. Behavioral targeting,
however, presents the most difficulty to XRay since it considers a large volume
of user-generated data. This presents difficulty to the shadow account manager
because the more data considered, the more shadow accounts are needed to
perform the differential correlation.

XRay's major breakthrough is only requiring a logarithmic number of shadow
accounts, rather than the na\"ive exponential number. However this still
constitutes a limitation: auditing just 51 emails still requires 21 shadow
accounts. This can be a serious limitation depending on the type of targeting
employed by the service provider and the volume of data generated by the user.
Since XRay considers shadow accounts to be ``generally scarce'' and limited to
``a couple dozen'', they must necessarily restrict their behavioral targeting to
``recent'' data.

With a peer-to-peer model where the shadow accounts are replaced by other real
accounts, this limitation may be unnecessary.

\section{Milestones}

\section{Challenges}

\section{Adjustments}

limitations of xray:

* shadow accounts are "generally scarce"
* assume users can obtain a "couple dozen" (despite "adversarial service reactions")
	* e.g. regular gmail accounts have a captcha, automated accounts either require a non-free custom domain or an education license
* limited-coverage: not every ad is seen in all relevant shadow accounts (just need more shadow accounts)
* overlapping-inputs: multiple inputs may trigger the same ad, dramatically reducing recall (solved by grouping "related" inputs acoording to the ads they are seen with)
* doesn't consider exclusion targeting
* gmail accuracy depends on personal assesment for ground truth
* sizes are grossly inaccurate (need 21 accounts for just 51 inputs)

limitations of crowdsourcing

* can't group inputs by contextual ad placement
* can't craft specific subsets of inputs to check for ads

0) Document in detail the progress you have made so far (note that this can be a part of your final report).
 
1) Measure your current progress against your project proposal, what milestones have you reached or missed? 
 
2) What challenges you have encountered were not anticipated in proposal? Have you addressed them (and how)? If not, how do you plan to address them?
 
3) What adjustment do you need to make to the proposal?

\end{document}
