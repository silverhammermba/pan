\documentclass{article}
\usepackage{amsmath,amssymb,enumerate,mathtools}
\usepackage[margin=1in]{geometry}

\title{CSE 450 Project Proposal:\\Crowdsourced Reverse Data Mining}
\author{Maxwell Anselm}
\date{\today}

\begin{document}
\maketitle

\section{Objectives}

Many web companies extensively employ data mining techniques to create
personalized web experiences. Targeted ads and product recommendations are
common examples of this. The prerequisite of such personalization
is access to personal data in order for the data mining algorithms to build a
profile of the user. Unfortunately, the volume of data required is so large
that---rather than asking users for specific targeting information---companies
simply mine {\it all} of the users' data (or as much as they can get their hands
on). This can be viewed as breach of privacy, since companies are not making
clear what personal data they are exploiting for the purpose of personalization,
and users are unaware of exactly how much information they are divulging by
using the service.

The goal of this project is to investigate various methods of uncovering which
data companies are mining from users in order to personalize their content.
Specifically, I will consider methods that utilize user collaboration in order
to build a body of data sufficient to accurately determine the mined information
based on the personalized content.

This will involve theoretical work in the feasibility and privacy-preservation
of such methods. It may also involve proof-of-concept code to determine
practicality of any methods found.

\section{Related Work}

This project is primarily inspired by XRay differential correlation tool. XRay
attempts to track personal data use on the web through the use of
locally-created fake accounts. The method is theoretically sound, but seems
impractical in real-world applications. This project seeks to bypass that
impracticality by replacing the fake accounts with (anonymized) real accounts
from other volunteer users.

\section{Plan}

I plan to first study the XRay tool in more detail. This project has a very
similar goal, so it is likely that several aspects of the XRay tool can be
applied as-is.

Afterwards I will attempt to prove how crowdsourcing can be applied to the
problem without comprosing user privacy while still allowing for differential
correlation.

Lastly (assuming success of the previous parts) I will try to build
proof-of-concept software to test the practical aspects. For example, whether
the computations can be done efficiently and the measurable accuracy of the
results.

\end{document}
